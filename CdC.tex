\documentclass[a4paper,11pt]{article}
\usepackage[T1]{fontenc}
\usepackage[utf8]{inputenc}
\usepackage{lmodern,textcomp}
\usepackage[french]{babel}
\usepackage[top=2cm, bottom=2cm, left=2cm, right=2cm]{geometry}
\title{Cahier des Charges de Projet\\PR-2101}
\author{Q. CHAUVIN, J. LEFEVRE, V. THAK}

\begin{document}

\maketitle
%\tableofcontents

\begin{abstract}
(Vision globale du projet)
L'objectif est de concevoir et de réaliser une machine de bureau recyclant les déchets plastiques\footnote{Uniquement ceux qui ne dégagent pas de gaz nocifs et qui sont utilisables dans l'impression 3D.} afin de produire du filament utilisables par les imprimantes 3D à "dépot de matières fondue". 
\end{abstract}

\section{Adaptation du projet sur un projet déjà existant}

FDM :technique consiste à faire fondre un filament de thermoplastique (généralement un plastique type ABS ou PLA) à travers une buse (ou extrudeur) chauffée à une température variant entre 160 et 400\textdegree  C suivant la température de plasticité du polymère. Le fil en fusion, d'un diamètre de l'ordre du dixième de millimètre , est déposé sur le modèle et vient se coller par re-fusion sur la couche précédente .

\section{Utilistation de solutions techniques déjà existantes}

\section{Présentation détaillé du projet}
\subsection{Descrition des contraintes}
\subsection{Planification des taches}
\section{Prise en charge des coûts}

\end{document}
